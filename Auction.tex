\documentclass[a4paper]{article}
\usepackage{indentfirst}
\usepackage[margin=3cm]{geometry}
\linespread{1.3}
\usepackage{tikz}
\usepackage{graphicx}
\usepackage{booktabs}
\usepackage{xcolor}
\usepackage{eurosym}
\usepackage{amsmath}
\begin{document}
\begin{center}
\LARGE{Microeconomics II--Auction}
\end{center}


\section*{An Auction--2015}
We consider the auction setting as discussed in class: 2 (risk neutral) bidders with values drawn independently from $U[0,1]$ and each bidder being privately informed about his own value $v_{i}$. The seller does not use a reserve price, he sells the object to the highest bidder, who has to pay the average of the two bids, hence, 
$$p = \frac{b_{1}+b_{2}}{2}$$
(The losing bidder does not have to pay.)
\begin{enumerate}
	\item (10 points) Determine a Bayesian Nash Equilibrium.
	\item   ( 5 points) Determine the seller's expected revenue.
	\item   ( 5 points) Determine the expected utility of a buyer with value $v_{i}$
	\item   ( 5 points) Assume that, in contrast to the situation above, the buyers (but not the seller) know each other's values. For given $(v_{1} v_{2})$, determine a Nash equilibrium. Calculate the seller's expected revenue when this equilibrium is played. (Hint: use the tie-breaking rule that, if $v_{i}>v_{j}$ and bidders make the same bid, the object is allocated to player i, that is, the player with the highest value.)
\end{enumerate}


\end{document}




